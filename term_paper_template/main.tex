%-----------------------------------------------------------------------------
% Template for seminar 'Program Analysis' at TU Darmstadt.
%
% Adapted from template for sigplanconf LaTeX Class, which is a LaTeX 2e
% class file for SIGPLAN conference proceedings (by Paul C.
% Anagnostopoulos).
%
%-----------------------------------------------------------------------------


\documentclass[authoryear,preprint]{sigplanconf}

% A couple of packages that may be useful
\usepackage{amsmath}
\usepackage{amsfonts}
\usepackage{amssymb}
\usepackage{amsthm}
\usepackage{algorithm2e}
\usepackage{listings}
\usepackage{xcolor}
\usepackage{tikz}
\usepackage{booktabs}
\usepackage{subfigure}
\usepackage[english]{babel}
\usepackage{blindtext}

\begin{document}

\special{papersize=a4}
\setlength{\pdfpageheight}{\paperheight}
\setlength{\pdfpagewidth}{\paperwidth}


\title{An approach to Contextual Transformation in EBSs}

\authorinfo{Suhas Chikkanna}{}{}

\maketitle




\section{Background}
\label{sec:Background}

Among Distributed Systems, the Event-based Systems(EBSs) are used to disseminate event
notifications on the occurrence of an event, generated by an event producer. The disseminated
event notification is delivered to the event consumer, which subscribes to an event of particular
interest. The event notifications are delivered to the event consumers based on a match/satisfying
criteria between the subscription and event notification. The EBSs are predominantly used in the
supply chain, new ticker and share market. These applications are widely distributed, with
heterogeneous components making up a system-wide architecture. An event detected \&/or generated
by a component, is propagated through event-notification service to the event consumer. The
current publish/subscribe systems utilizes a common context, to infer an event produced by an event
producer and, subscription subscribed by an event consumer. The producer and consumers may
often differ in their respective contexts. And the differing contexts between producers and
consumers leads to a difference in the interpretation of the event. Below in the related work section,
ACTrESS suggests a solution to this differing context.


\section{Related Work}
\label{sec:Related Work}

ACTrESS suggests automatic contextual transformation for differing contexts in Event-Based
Software Systems at runtime, transparent to end users and also, ACTrESS overcomes major
challenges related to contextual transformations. In ACTrESS, the Producer Context is transformed
to Root Context (at the broker the producer/publisher is connected to) and the Root Context is
transformed to Consumer Context at the fringe of the broker network(i.e., at last broker the
consumer is connected to).

\section{Problem Statement}
\label{sec:Problem Statement}

The ACTrESS prototype implements greedy algorithm and works best in most cases. But in
ACTrESS, there may result unnecessary work in spite of producer and consumer being in the same
context. For instance, in a channel-based pub/sub system and Topic-based pub/sub system. And also, certain topologies such as mobile networks where the clients(i.e.,
producer/consumer) move from one broker to another, the same transformation (Producer Context
→ Root Context → Consumer Context) performed in the former broker has to be performed again
on the latter broker, where the client may newly join. We try to address this with our heuristic in the
below section.

\section{Our Approach}
\label{sec:Our Approach}

Our approach focuses on, contextual transformation with respect to taking into account the number of transformations to be performed and the number of messages generated in the network due to the transformations. Our approach proposes to perform contextual transformation, at the first broker with a subscriber in different context. Below listed are the different cases in order to illustrate our approach.


\subsection{Citations}

Use citations to refer to other 
papers~\cite{HerlihyMoss1993-TransactionalMemory,FraserHanson1992-CodeGenerator} 
and books~\cite{Strunk-ElementsOfStyle,Aho86-Compilers}.


\subsection{Tables}

Table~\ref{t:Translations} shows how a table looks like.

\begin{table}[ht]
\centering
\begin{tabular}{ll}
\hline
\textbf{English} & \textbf{German}\\
\hline
cell phone       & Handy\\
Diet Coke        & Coca Cola light\\
\hline
\end{tabular}
\caption[Translations]{\label{t:Translations}Translations.}
\end{table}

\subsection{Figures}

Figure~\ref{f:SOLAlogo} shows a simple figure with a single picture
and Figure~\ref{f:SubfigureExample} shows a more complex figure
containing subfigures.

\begin{figure}[ht]
\centering
\includegraphics[width=.6\linewidth]{figures/SOLALogo}
\caption[SOLA logo]{\label{f:SOLAlogo}SOLA logo.}
\end{figure}

\begin{figure}[ht]
\centering
\subfigure[TUDaLogo]{\includegraphics[height=12mm]{figures/TUDaLogo}}\quad
\subfigure[SOLALogo]{\includegraphics[height=12mm]{figures/SOLALogo}}
\caption[Subfigure example]{\label{f:SubfigureExample}Two pictures as
  part of a single figure through the magic of the subfigure package.}
\end{figure}


\subsection{Source code}

The listings package provides tools to typeset source code
listings. It supports many programming languages and provides a lot of
formatting options.

\lstset{numbers=left, numberstyle=\tiny, stepnumber=1, numbersep=5pt}
\lstset{basicstyle=\ttfamily}
\lstset{frame=tb}

\begin{lstlisting}[float,caption=Example usage of the listing package,label=l:javaClass,language=Java]
class S {
   int f1 = 42;
   public S(int x) {
          f1 = x;
   }
}
\end{lstlisting}

Listing \ref{l:javaClass} shows an example listing. Code snippets can
also be inserted in normal text:
\verb$\lstinline|int f1 = 42;|$ gives \lstinline$int f1 = 42;$


\subsection{Miscellany}

\begin{description}

\item[Capitalization.] When referring to a named table (such as in the
  previous section), the word \emph{table} is capitalized. The same is
  true for figures, chapters and sections.

\item[Bibliography.] Use \verb|bibtex| to make your life easier and to
  produce consistently formatted entries.

\item[Contractions.] Avoid contractions. For instance, use ``do not''
  rather than ``don't.''

\item[Style guide.] A classic reference book on writing style is
  Strunk's \emph{The Elements of Style} \cite{Strunk-ElementsOfStyle}.

\end{description}


\section{Another Section}

\blindtext % replace this with your own text


\section{Yet Another Section}

\blindtext % replace this with your own text


\section{Conclusion}

\blindtext % replace this with your own text

\bibliographystyle{abbrvnat}
\bibliography{references}


\bibliographystyle{abbrvnat}



\end{document}
